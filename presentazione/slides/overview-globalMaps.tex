\iffalse
Una della particolarità di questo linguaggio è la presenza di tre tipi di mappe globali. Tali mappe permettono di salvare dei valori e di accedervi ovunque nel codice.
La prima è la heap map, che viene costruita sullo heap. Per questo tipo di mappa in caso di predicati che prevedono il backtracking il risultato viene aggiornato
La seconda e la terza invece sono global map e la table map, che vengono create in un area di memoria globale dove picat viene lanciato. La differenza tra i due è che la table map utilizza l'hash consing, ovvero al posto di memorizzare valori uguali ne viene solamente salvato un riferimento, in modo da avere un miglioramente di prestazioni in tempo e spazio
È possibile creare differenti istanze di queste mappe globali associando ad ognuna di esse un identificativo, con il quale anche è possibile accedervi
\fi

\begin{frame}[fragile, shrink=1]{Overview - Global maps}

	Presenza di tre tipi di mappe globali:
	\begin{itemize}
		\item heap map
		\item global map \tikzmark{a}
		\item table map\hspace{0.15cm}  \tikzmark{b}
	\end{itemize}
\mybrace{a}{b}[No aggiornamento in caso di backtracking]

	\begin{lstlisting}
%Esempio di utilizzo
fun() =>
	X = get_global_map(0),
	println(X.get(2)),
	X.put(1, val1modified).
main =>
	X = get_global_map(0),
	X.put(1, val1),X.put(2,val2),X.put(3,val3),
	fun(),
	println(X.get(1)).
	\end{lstlisting}
\end{frame}