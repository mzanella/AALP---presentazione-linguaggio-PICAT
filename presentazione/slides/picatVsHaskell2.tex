\iffalse
Qui possiamo vedere un esempio di due programmi uno scritto in PICAT, l'altro in haskell, che ordinano delle liste utilizzando il quicksort. Come possiamo notare grazie al pattern matching e alla list comprehension i due codici sono pressochè uguali.
\fi

\begin{frame}[fragile, shrink=1]{PICAT vs Haskell - 2}
	Quicksort in Picat							
	\begin{lstlisting}
quicksort([]) = [].							
quicksort([H|T]) = L ++ [H] ++ R => 		
   L = quicksort([E : E in T, E <= H]),	
   R = quicksort([E : E in T, E > H]).		
	\end{lstlisting}
\vspace{1cm}
	Quicksort in Haskell
	\begin{lstlisting}
quicksort [] = []
quicksort (x:xs) = l ++ [x] ++ r
   where l = quicksort [z | z <- xs, z <= x]
         r = quicksort [z | z <- xs, z > x]
	\end{lstlisting}
\end{frame}

