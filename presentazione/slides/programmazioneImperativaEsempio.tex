\iffalse
Qui possiamo vedere un esempio di programma in stile imperativo scritto in PICAT. Il programma inizia dal main con una lista vuota X e un numero I pari a zero. Poi viene fatto un wile inizializzando la lista e incrementando ad ogni ciclo I. Finito il ciclo viene richiamata la funzione prova modifica la lista.
Poi la lista viene riassegnata e ancora modificata con la funzione prova
\fi

\begin{frame}[fragile, shrink=20]{Programmazione imperativa - Esempio}

	\lstinputlisting{../examples/imperativeProgramming.pi}

\end{frame}