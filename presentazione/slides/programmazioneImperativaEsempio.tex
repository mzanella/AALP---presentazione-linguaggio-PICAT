\iffalse
Qui possiamo vedere un esempio di programma in stile imperativo scritto in PICAT. Il programma inizia dal main con una lista vuota X e un numero I pari a zero. Poi viene fatto un wile inizializzando la lista e incrementando ad ogni ciclo I. Finito il ciclo viene richiamata la funzione prova modifica la lista.
Poi la lista viene riassegnata e ancora modificata con la funzione prova
\fi

\begin{frame}[fragile, shrink=20]{Programmazione imperativa - Esempio}

	\begin{lstlisting}
% Example of imperative programming in Picat
prova(X) =>
	if (X.length <= 3) then
		X[1] := -1
	else
		foreach (I in 1..X.length)
			X[I] := X[I] - 2
		end
	end.
main =>
	X := [],
	I := 0, 
	while (I <= 5)
		X := X ++ [I],
		I := I + 1
	end,
	prova(X), println(X),
	X := [1,2,3], prova(X), println(X).
	\end{lstlisting}
	
\end{frame}